\documentclass[mat1, tisk]{fmfdelo}
% \documentclass[fin1, tisk]{fmfdelo}
% Če pobrišete možnost tisk, bodo povezave obarvane,
% na začetku pa ne bo praznih strani po naslovu, …

%%%%%%%%%%%%%%%%%%%%%%%%%%%%%%%%%%%%%%%%%%%%%%%%%%%%%%%%%%%%%%%%%%%%%%%%%%%%%%%
% METAPODATKI
%%%%%%%%%%%%%%%%%%%%%%%%%%%%%%%%%%%%%%%%%%%%%%%%%%%%%%%%%%%%%%%%%%%%%%%%%%%%%%%

% - vaše ime
\avtor{Matija Kocbek}

% - naslov dela v slovenščini
\naslov{Redkejši grafi z velikim kromatičnim številom}

% - naslov dela v angleščini
\title{Sparse graphs with high chromatic number}

% - ime mentorja/mentorice s polnim nazivom:
%   - doc.~dr.~Ime Priimek
%   - izr.~prof.~dr.~Ime Priimek
%   - prof.~dr.~Ime Priimek
%   za druge variante uporabite ustrezne ukaze
\mentor{prof.~dr.~Riste Škrekovski}
% \somentor{...}
% \mentorica{...}
% \somentorica{...}
% \mentorja{...}{...}
% \somentorja{...}{...}
% \mentorici{...}{...}
% \somentorici{...}{...}

% - leto diplome
\letnica{2025} 

% - povzetek v slovenščini
%   V povzetku na kratko opišite vsebinske rezultate dela. Sem ne sodi razlaga
%   organizacije dela, torej v katerem razdelku je kaj, pač pa le opis vsebine.
\povzetek{...}

% - povzetek v angleščini
\abstract{...}

% - klasifikacijske oznake, ločene z vejicami
%   Oznake, ki opisujejo področje dela, so dostopne na strani https://www.ams.org/msc/
\klasifikacija{..., ...}

% - ključne besede, ki nastopajo v delu, ločene s \sep
\kljucnebesede{...\sep ...}

% - angleški prevod ključnih besed
\keywords{...\sep ...} % angleški prevod ključnih besed

% - angleško-slovenski slovar strokovnih izrazov
\slovar{
% \geslo{angleški izraz}{slovenski izraz}
% ...
}

% - ime datoteke z viri (vključno s končnico .bib), če uporabljate BibTeX
% \literatura{....bib}

%%%%%%%%%%%%%%%%%%%%%%%%%%%%%%%%%%%%%%%%%%%%%%%%%%%%%%%%%%%%%%%%%%%%%%%%%%%%%%%
% DODATNE DEFINICIJE
%%%%%%%%%%%%%%%%%%%%%%%%%%%%%%%%%%%%%%%%%%%%%%%%%%%%%%%%%%%%%%%%%%%%%%%%%%%%%%%

% naložite dodatne pakete, ki jih potrebujete
% \usepackage{...}

% deklarirajte vse matematične operatorje, da jih bo LaTeX pravilno stavil
% \DeclareMathOperator{\...}{...}

% vstavite svoje definicije ...
% \newcommand{\...}{...}

%%%%%%%%%%%%%%%%%%%%%%%%%%%%%%%%%%%%%%%%%%%%%%%%%%%%%%%%%%%%%%%%%%%%%%%%%%%%%%%
% ZAČETEK VSEBINE
%%%%%%%%%%%%%%%%%%%%%%%%%%%%%%%%%%%%%%%%%%%%%%%%%%%%%%%%%%%%%%%%%%%%%%%%%%%%%%%

\begin{document}

\section{Uvod}
Grafi ponazarjajo relacije med različnimi objekti. Kromatično število nam pove, kako težko je razbiti graf na množice vozlišč, ki med seboj nimajo nobene povezave.
Z drugimi besedami, pove nam, kakšno je minimalno število neodvisnih množic, ki jih potrebujemo, da lahko graf razbijemo na neodvisne množice.

    \begin{definicija}
        Naj bo $G = (V, E)$ (neusmerjen) graf. Naj bo $K$ poljubna neprazna množica moči $k$. Tedaj preslikavi $c: V \to K$ pravimo 
        $k$-barvanje vozlišč grafa $G$. Če sta poljubni sosednji vozlišči v $G$ pobarvani z različnimi barvami, tj. če za sosednji 
        $u$ in $v$ velja, da je $c(u) \neq c(v)$, potem pravimo, da je takšno $k$-barvanje dobro. Kromatično število $G$ je najmanjše
        število $k$, za katerega obstaja dobro $k$-barvanje grafa $G$. Označimo ga z $\chi(G)$.
    \end{definicija}

    \begin{definicija}
        Naj bo $G = (V, E)$ in $H \subseteq V$. Če nobeni dve vozlišči iz $H$ nista med seboj povezani, pravimo, da je $H$ neodvisna množica. Velikost neodvisne množice
        z največjo močjo imenujemo neodvisnostno število grafa $G$ in ga označimo z $\alpha(G)$.
    \end{definicija}

Kromatično in neodvisnostno število povezuje naslednja osnovna neenakost:
    
    \begin{trditev}
        Za vsak graf $G$ na $n$ vozliščih velja $\alpha(G)\chi(G) \geq n$.
        \label{spodnja_meja}
    \end{trditev}

    \begin{dokaz}
        Označimo z $V_i$ množico vozlišč, ki jih neko optimalno dobro barvanje pobarva z $i$-to barvo. Iz definicije dobrega barvanja direktno sledi, da nobeni dve vozlišči
        iz $V_i$ ne moreta biti povezani, saj sta enako pobarvani. Torej je $V_i$ neodvisna množica. To pa pomeni, da je $|V_i| \leq \alpha(G)$. Torej je 
        $n = \sum_{i=1}^{\chi(G)}|V_i| \leq \sum_{i=1}^{\chi(G)}\alpha(G) = \chi(G)\alpha(G)$ in je s tem trditev dokazana.
    \end{dokaz}

Ta trditev je pomembna, saj je to ena od redkih znanih spodnjih mej za kromatično število. Velja tudi naslednja najenostavnejša zgornja meja za kromatično število:

    \begin{trditev}
        Označimo z $\Delta(G)$ maksimalno stopnjo vseh vozlišč v grafu $G$. Tedaj je $\chi(G) \leq \Delta(G) + 1$
        \label{zgornja_meja}
    \end{trditev}

    \begin{dokaz}
        PREPIŠI, KAKO JE KONVALINKA V KNJIGI ZAPISAL TA DOKAZ S POŽREŠNIM BARVANJEM.
    \end{dokaz}

Za grafe brez trikotnikov velja naslednja karakterizacija:

    \begin{trditev}
        Graf $G$ nima trikotnikov natanko tedaj, ko je za vsako vozlišče $v$, njegova množica sosedov $N(v)$ neodvisna množica.
        \label{karakterizacija}
    \end{trditev}

    \begin{dokaz}
        Naj bo $G$ brez trikotnikov. Naj bo $v$ poljubno vozlišče. Denimo, da sta vozlišči $u$ in $w$ iz $N(v)$ povezani. Tedaj $v, u$ in $w$ tvorijo trikotnik, kar je protislovje.

        Naj bo za vsako vozlišče $v$ množica $N(v)$ neodvisna. Denimo, da imamo trikotnik sestavljen iz vozlišč $u, v$ in $w$. Tedaj sta $v, w \in N(u)$, hkrati pa sta $v$ in $w$ povezani,
        kar je v protislovju z neodvisnostjo $N(u)$. 
        
        S tem smo dokazali obe smeri karakterizacije.
    \end{dokaz}

Tema tega diplomskega dela so grafi, ki imajo veliko ožino tj. velikost najmanjšega cikla. 

    \begin{definicija}
        Dolžini največjega cikla v grafu $G$ pravimo ožina in jo označimo z $girth(G)$.
    \end{definicija}

Takšni grafi so lokalno izredno enostavni. Če je ožina grafa $g$, bo podgraf porojen s poljubnih $g - 1$ vozlišč dejansko gozd, ker ne more 
imeti ciklov. Pokazali pa bomo, da so lahko takšni grafi globalno zelo kompleksni, če za mero kompleksnosti vzamemo kromatično število. 
Videli bomo, da ima lahko graf s poljubno veliko ožino prav tako poljubno veliko kromatično število. Razvoj razumevanja takšnih grafov 
bomo predstavili skozi primere.

\section{Grafi brez trikotnikov s poljubno velikim kromatičnim številom}
Najmanjši možen cikel v grafu je cikel dolžine tri in takšnim ciklom pravimo trikotniki. Če nas zanima, ali lahko graf s poljubno veliko
ožino ima poljubno veliko kromatično število, moramo začeti na prvem koraku in se vprašati, ali lahko ima sploh graf brez trikotnikov
poljubno veliko ožino. Pokazali bomo, da lahko, s tem da bomo predstavili nekaj konstrukcij takšnih grafov.

TU ALI PA NA KONCU RAZDELKA DODAJ TRDITEV IN DOKAZ, DA SO SKORAJ VSI GRAFI BREZ TRIKOTNIKOV DEJANSKO DVODELNI GRAFI (V mapi imaš članek
"Almost all triangle-free graphs are bipartite") IN IMAJO KOT TAKŠNI KROMATIČNO ŠTEVILO 2 IN DA ZATO REZULTATI IZ TEGA RAZDELKA RES NISO 
OČITNI!!!!!!! DODAJ PRAV TAKO KARAKTERIZACIJO DVODELNIH GRAFOV KOT GRAFOV BREZ LIHIH CIKLOV!!!!!

\subsection{Tuttova konstrukcija}
Prvi, ki je pokazal, da obstajajo grafi brez trikotnikov, ki imajo poljubno veliko kromatično število, je bil William Thomas Tutte, ki je
pisal pod psevdonimom Blanche Descartes. Podal je sledečo induktivno konstrukcijo.

Indukcijo delamo na kromatičnem številu $k$. Začnemo z grafom $G_1$, ki vsebuje samo eno vozlišče. Denimo sedaj, da poznamo graf $G_k$, 
ki ima $n$ vozlišč. Graf $G_{k+1}$ zgradimo tako, da vzamemo množico $Y$ z $k(n - 1) + 1$ vozlišči in brez kakršnihkoli povezav med njimi. 
Za vsak $X \subseteq Y$, ki vsebuje $n$ vozlišč vzamemo kopijo grafa $G_k$, ki jo označimo z $G_X$ in povežemo to kopijo z $X$ tako, da je 
vsako vozlišče iz $X$ povezano z natanko enim vozliščem iz $G_X$. Različnih kopij grafa $G_k$ med seboj ne povezujemo, tj. med nobenima 
vozliščema iz $G_X$ in $G_{X'}$ ne obstaja povezava, če je $X \neq X'$. S tem smo zgradili $G_{k+1}$.

Pokažimo, da je $\chi(G_k) = k$ in da $G_k$ ne vsebuje trikotnikov.

    \begin{trditev}
        Graf $G_k$ iz Tuttove konstrukcije ne vsebuje nobenega trikotnika in velja $\chi(G_k) = k$.
    \end{trditev}
    
    \begin{dokaz}
        Dokazujemo z indukcijo. Graf z enim vozliščem ima kromatično število $1$ in je brez trikotnikov, torej je baza indukcije izpoljnena. 
        Denimo, da trditev za $G_k$ in pokažimo, da velja za $G_{k+1}$.

        Najprej s protislovjem pokažimo, da je $\chi(G_{k+1}) \geq k + 1$. Denimo nasprotno, da je $\chi(G_{k+1}) \leq k$. Tedaj obstaja $c$, 
        ki je pravilno $k$-barvanje grafa $g_{k+1}$. Recimo, da $G_k$ ima $n$ vozlišč. Tedaj po konstrukciji velja, da ima množica $Y$ iz 
        konstrukcije $k(n - 1) + 1$ vozlišč, ki so pobarvana pravilno z večjemu $k$ različnimi barvami. Zato mora obstajati vsaj ena barva 
        $b$, s katero je pobarvanih vsaj $n$ vozlišč iz $Y$. Vzemimo poljubnih $n$ vozlišč iz $Y$ pobarvanih s $b$ in označimo to množico 
        z $X$. Ker je po indukcijski predpostavki $\chi(G_k) \geq k$, bo $c$ pobarval $G_X$, ki je kopija $G_k$ z natanko $k$ barvami, saj 
        je $c$ pravilno barvanje. Ker je vsako vozlišče v $X$ povezano z natanko enim v $G_X$ in ker je $c$ pravilno barvanje, je $b$ različna
        barva od vseh $k$ barv, s katerimi smo pobarvali $G_X$. Torej je $c$ pobarval $G_{k+1}$ z vsaj $k + 1$ barvami, kar pa je protislovje.

        Prav tako je $\chi(G_{k+1}) \leq k + 1$. Po indukcijski predpostavki lahko namreč vsako kopijo $G_k$ pobarvamo s $k$ barvami, saj različne
        kopije niso med seboj povezane. Tedaj lahko vzamemo poljubno novo barvo, ki je nismo uporabili za kopije $G_k$ in z njo pobarvamo $Y$,
        saj elementi $Y$ nimajo med seboj povezav. S tem smo dokazali želeno, saj smo dobili pravilno $(k+1)$-barvanje grafa, in je res 
        $\chi(G_{k+1}) = k + 1$.

        Denimo, da v $G_{k+1}$ obstaja trikotnik. V trikotniku so vsa vozlišča paroma povezana, zato lahko v njem leži kvečjemu eno vozlišče iz $Y$,
        saj med vozlišči v $Y$ ni povezav. Prav tako preostali dve vozlišči morata ležati v isti kopiji $G_k$, saj nimamo povezav med različnimi kopijami.
        To pa pomeni, da imamo vozlišče v $Y$, ki je povezano z dvema različnima vozliščema iz iste kopije $G_k$, kar pa je v protislovju s konstrukcijo.
        Torej $G_{k+1}$ res ne vsebuje trikotnikov.

    \end{dokaz}

    \begin{opomba}
        Če definiramo $G_k$ le za $k \geq 3$ in za $G_3$ vzamemo cikel dolžine sedem, ob zgornji trditvi velja celo, da je $girth(G_k) \geq 6$ za vsak 
        $k \geq 3$. DOKAŽI TO!!!!!!!!!
    \end{opomba}

Tuttova konstrukcija torej res dokazuje, da lahko imajo grafi brez trikotnikov poljubno veliko kromatično število. Vendar so grafi v Tuttejevi konstrukciji
izredno veliki (če ima $G_k$ recimo $n$ vozlišč, ima $G_{k+1}$ potem $\binom{k(n-1)+1}{n}n + k(n-1) + 1$ vozlišč) in je v resnici kromatično število zelo majhno 
v razmerju s številom vozlišč. Iz tega aspekta je bolj zanimiva konstrukcija Mycielskega. Glavna prednost Tuttejeve konstrukcije je ta, da dokazuje, da lahko imajo 
tudi grafi brez ciklov dolžine manjše ali enake $4$ ali $5$ poljubno velika kromatična števila.

\subsection{Konstrukcija Mycielskega}
Jan Mycielski je leta 1955 podal konstrukcijo, ki iz začetnega grafa z $n$ vozlišči zgradi graf z $2n + 1$ vozlišči, ki ima večje kromatično število kot
začetni graf, hkrati pa nima trikotnikov, če jih začetni graf nima. Konstrukcija je podana na naslednji način.

Denimo, da imamo graf $G$ na $n$ vozliščih ${v_1, \ldots, v_n}$. Potem definiramo $M(G)$ kot graf z $2n + 1$ vozlišči ${a_1, \ldots, a_n, b_1, \ldots, b_n, c}$. 
Za vse $i, j$, za katere obstaja povezava $v_iv_j$ v $G$, tvorimo povezave $a_ia_j, a_ib_j$ in $a_jb_i$ v $M(G)$. Ob tem za vsak $i$ med $1$ in $n$ tvorimo 
povezavo $b_ic$ v $M(G)$. Takšnemu grafu $M(G)$ pravimo graf Mycielskega grafa $G$.

    \begin{trditev}
        Če graf $G$ nima trikotnikov, potem nima trikotnikov niti njegov graf Mycielskega $M(G)$.
    \end{trditev}

    \begin{dokaz}
        Naj bo $G$ brez trikotnikov. Dokazujemo s protislovjem. Denimo, da ima $M(G)$ nek trikotnik. Vsa vozlišča znotraj trikotnika so med seboj povezana. Ker v $M(G)$ ni povezav med 
        $b_i$ in $b_j$ za nobena $i$ in $j$, je lahko v ciklu kvečjemu eno vozlišče oblike $b_i$. To pomeni, da mora biti vsaj eno vozlišče oblike $a_i$ v trikotniku.
        Ker $c$ ni povezan z nobenim vozliščem te oblike, $c$ ne more biti v trikotniku. Torej imamo v trikotniku $a_j$ in $a_k$ za neka različna $j$ in $k$. Če bi imeli v
        trikotniku še vozlišče $a_i$ za nek $i$ različen od $j$ in $k$, bi to pomenilo, da imamo trikotnik v $G$, saj je podgraf $M(G)$ porojen z vozlišči ${a_1, \ldots, a_n}$
        izomorfen $G$ po konstrukciji, kar je protislovje. Torej je v trikotniku še vozlišče $b_i$ za nek $i$. To pa po konstrukciji $M(G)$ pomeni, da imamo v $G$ povezave $v_iv_k$, $v_iv_j$
        in $v_jv_k$, kar je trikotnik. Ker $G$ po predpostavki nima trikotnikov, smo prišli do protislovja.
    \end{dokaz}

    \begin{trditev}
        Velja $\chi(M(G)) = \chi(G) + 1$.
    \end{trditev}

    \begin{dokaz}
        Pokažimo najprej, da je $\chi(M(G)) \geq \chi(G) + 1$. Dokazujemo s protislovjem. Denimo, da je $\chi(M(G)) \leq \chi(G) = k$. Tedaj obstaja pravilno $k$-barvanje $M(G)$, recimo mu $f$.
        Brez škode za splošnost je $f(c) = k$. Zaradi pravilnosti $f$, ni nobeno vozlišče oblike $b_i$ pobarvano s $k$. Barvanje $f$ porodi pravilno $k$-barvanje grafa $G$, recimo mu $g$, podano 
        z $g(v_i) = f(a_i)$. Če je kakšno vozlišče $v_i$ v $G$ pobarvano s $k$, lahko spremenimo barvo v $f(b_i)$ in je barvanje grafa $G$ še vedno pravilno. Namreč, če imamo povezavo $v_iv_j$ v $G$, 
        imamo tudi povezavo $b_ia_j$ v $M(G)$, kar pomeni, da je $f(b_i) \neq f(a_j) = g(a_j)$ zaradi pravilnosti barvanja $f$. Torej tudi v spremenjenem barvanju nimamo nobenih sosedov z enako barvo,
        torej je to pravilno barvanje $G$. Vsa vozlišča v $G$, ki so bila pobarvana s $k$, smo na novo pobarvali z neko barvo iz $\{1, \ldots, k-1\}$, ker je $f(b_i) \neq k$ za vse $i$. To pa pomeni,
        da smo našli pravilno $(k-1)$-barvanje $G$, kar je v protislovju z $\chi(G) = k$.
        Dokažimo še $\chi(M(G)) \leq \chi(G) + 1$. Naj bo $k = \chi(G)$ in naj bo $g$ pravilno $k$-barvanje grafa $G$. Definiramo potem $f$ kot $(k+1)$-barvanje grafa $M(G)$. Naj bo $f(a_i) = f(b_i) = g(v_i)$
        za vse $i$. Naj bo $f(c) = k + 1$. Pokažimo, da je $f$ pravilno barvanje. Ker je $f(b_i) \neq k + 1$ za vse $i$, $c$ nima enake barve z nobenim sosedom. Če pa imamo v $M(G)$ povezavo oblike $a_ia_j$ ali 
        pa $a_ib_j$ za neka različna $i$ in $j$, vemo, da imamo v $G$ povezavo $v_iv_j$. Ker je $f(a_i) = g(v_i) \neq g(v_j) = f(a_j) = f(b_j)$, pri čemer smo upoštevali pravilnost $g$, vemo, da niti sosedi oblike
        $a_i$ in $a_j$ ali $a_i$ in $b_j$ ne bodo imeli enake barve. Torej je $f$ res pravilno $(k+1)$-barvanje $M(G)$ in je res $\chi(M(G)) = \chi(G) + 1$.
    \end{dokaz}

    \begin{posledica}
        Naj bo $G_3$ cikel dolžine $5$. Naj bo $G_{k+1} = M(G_k)$ za vsak $k \geq 3$. Tedaj je $G_k$ brez trikotnikov in $\chi(G_k) = k$ za vsak $k \geq 3$.
    \end{posledica}

    \begin{dokaz}
        Dokažemo z indukcijo. $G_3$ je cikel lihe dolžine, torej ima kromatično število $3$, in nima trikotnikov, saj cikel dolžine $3$. Baza indukcije je s tem izpolnjena. Indukcijski korak dokažemo enostavno 
        s prejšnjima trditvama.
    \end{dokaz}

Konstrukcija Mycielskega nam torej porodi še eno družino grafov brez trikotnikov, katerih kromatično število neomejeno narašča. Grafi iz te družine so dosti manjši kot grafi iz Tuttejeve konstrukcije,
in sicer bi z indukcijo lahko enostavno dokazali, da je število vozlišč v $G_k$ enako $3 \cdot 2^{k-2} - 1$ za $k \geq 3$. Torej kromatično število narašča logaritemsko glede na število vozlišč. To je še vedno dosti
majhno, vendar veliko boljše kot v Tuttovi konstrukciji.

\subsection{Zamični grafi}

\subsection{Ramseyjevi grafi}
Vsi grafi iz prejšnjih podrazdelkov so dejansko imeli precej majhna kromatična števila glede na velikosti grafov. V tem podrazdelku pa bomo obravnavali grafe z dosti večjim kromatičnim številom,
torej globalno dosti bolj kompleksne grafe. Poglejmo naslednjo družino grafov.

Imejmo neko naravno število $m$ in geometrijo $\Gamma$ s točkami $\mathcal{P}$ in premicami $\mathcal{L}$ s poljubno incidenčno relacijo, za katero velja aksiom $\mathcal{A}$, da skozi $m-1$ paroma različnih točk poteka kvečjemu ena premica. Naj bo $A$ končna podmnožica $\mathcal{P}$ in
$B$ končna podmnožica $\mathcal{L}$. Tedaj lahko $A$ uredimo s poljubno linearno urejenostjo $<$. Nato tvorimo graf $\mathcal{G}$ tako, da za vozlišča vzamemo vse urejene pare točk in premic $(p, L), p \in A, L \in B$, za katere velja,
da $p$ leži na $L$. Tem parom rečemo incidenčni pari. Vozlišči $(p, L)$ in $(p', L')$ povežemo, če velja, da je $p < p'$, da sta $L$ in $L'$ različni in da $p$ leži na $L'$. Tedaj veljajo naslednje trditve:

    \begin{trditev}
        $\mathcal{G}$ ne vsebuje klike velikosti $m$.
    \end{trditev}

    \begin{dokaz}
        Dokazujemo s protislovjem. Denimo, da imamo kliko $(p_1, L_1), (p_2, L_2), \ldots,\\ (p_m, L_m)$. Tedaj so vse točke in premice paroma različne. Naj bo brez škode za splošnost $p_1 < p_2 < \ldots < p_m$. Tedaj po konstrukciji
        velja, da $p_1, \ldots p_{m-1}$ ležijo na $L_{m-1}$ in $L_m$. Ker imamo geometrijo, v kateri skozi $m-1$ paroma različnih točk poteka kvečjemu ena premica, je $L_{m-1} = L_m$, kar pa je protislovje. Torej $\mathcal{G}$ res ne vsebuje klike velikosti $m$.
    \end{dokaz}

    \begin{trditev}
        Velja $\alpha(\mathcal{G}) \leq |A| + |B|$.
    \end{trditev}

    \begin{dokaz}
        Naj bo $H \subseteq V(\mathcal{G})$ poljubna neodvisna množica vozlišč. Tedaj lahko $H$ zapišemo kot disjunktno unijo dveh množic.

        Prva množica vozlišč vsebuje urejene pare točk in premic iz $H$, za katere velja, da se točka pojavi v vsaj dveh urejenih parih iz $H$. Vzemimo dva takšna urejena para, v katerih se točka ponovi, recimo $(p, L)$ in $(p, L')$.
        Ker je množica $H$ neodvisna, mora za vsako vozlišče $(q, L') \in H, q \neq p$ veljati $q < p$, saj bi sicer imeli povezavo med $(p, L)$ in $(q, L')$, kar bi bilo v protislovju z neodvisnostjo $H$. Podobno mora 
        za vsako vozlišče $(q, L) \in H, q \neq p$ veljati $q < p$, saj bi sicer imeli povezavo med $(p, L')$ in $(q, L)$. To pa že pomeni, da je izmed vseh točk, ki se pojavijo v paru z $L$ v množici $H$, točka $p$ maksimalna.
        Ker je maksimum v linearni urejenosti enolično določen, se bo z vsake premice, ki se pojavi v nekem paru v $H$, kvečjemu ena točka pojavila v več kot enem paru. To pomeni, da bo v prvi množici vozlišč vsaka premica iz
        $B$ nastopila v kvečjemu enem paru. V prvi množici bo torej kvečjemu $|B|$ vozlišč.

        Druga množica vozlišč vsebuje urejene pare točk in premic iz $H$, za katere velja, da se točka pojavi v natanko enem urejenem paru iz $H$. Vozlišč v drugi množici bo torej kvečjemu toliko, kolikor je točk v $A$, torej kvečjemu $|A|$.
        
        Množica $H$, ki je unija prve in druge množice, vsebuje torej kvečjemu $|A| + |B|$ vozlišč. Ker je bila $H$ poljubna neodvisna množica, smo dokazali $\alpha(\mathcal{G}) \leq |A| + |B|$.
    \end{dokaz}

    \begin{trditev}
        Naj bo $L_A(\Gamma)$ množica točk iz $A$, ki leži na $L \in B$ v geometriji $\Gamma$. Velja $\chi(\mathcal{G}) \geq \frac{\sum_{L \in B}|L_A(\Gamma)|}{|A| + |B|}$.
    \end{trditev}

    \begin{dokaz}
        Graf $\mathcal{G}$ ima ravno $\sum_{L \in B}|L_A(\Gamma)|$ vozlišč. Če upoštevamo prejšnjo trditev in neenakost \ref{spodnja_meja}, dobimo, da je res $\chi(\mathcal{G}) \geq \frac{\sum_{L \in B}|L_A(\Gamma)|}{|A| + |B|}$.
    \end{dokaz}

Osredotočimo se zdaj na primer, ko je $m = 3$, torej na primer, ko skozi poljubni različni točki v geometriji poteka kvečjemu ena premica in ko graf $\mathcal{G}$ ne vsebuje trikotnikov. Zanima nas, kako optimalno izbrati $\mathcal{P}, \mathcal{L}, A$ in $B$, da bo $\mathcal{G}$ imel
čim manjše neodvisnostno število in posledično čim večje kromatično število. Povejmo to malo bolj formalno. Denimo, da je dano naravno število $n$. Denimo nadalje, da imamo podano množico točk $\mathcal{P}$ in premic $\mathcal{L}$. Naj potem množica $\mathcal{N}_{(\mathcal{P}, \mathcal{L})}$
vsebuje vse nabore točk in premic, za katere je število incidenčnih parov med temi točkami in premicami enako $n$, torej za katere je $|V(\mathcal{G})| = n$. Natančneje, naj bo $\mathcal{N}_{(\mathcal{P}, \mathcal{L})} = \{(A, B); A \subseteq \mathcal{P}, B \subseteq \mathcal{L}, \sum_{L\in B}|L_A(\Gamma)| = n\}$. 
Naj bo nadalje  $m_{(\mathcal{P}, \mathcal{L})}$ najmanjši nabor točk in premic v geometriji podani s $(\mathcal{P}, \mathcal{L})$, med katerimi je natanko $n$ incidenčnih parov, tj. naj bo $m_{(\mathcal{P}, \mathcal{L})} = \min\{|A|+|B|; (A, B) \in \mathcal{N}_{(\mathcal{P}, \mathcal{L})}\}$. 
Če imamo fiksno geometrijo $(\mathcal{P}, \mathcal{L})$, bomo ob optimalni izbiri $A$ in $B$ po drugi izmed prejšnjih trditev dobili graf s številom vozlišč $n$ in z neodvisnostnim številom $m_{(\mathcal{P}, \mathcal{L})}$. Označimo za konec še 
$C(n) = \min\{m_{(\mathcal{P}, \mathcal{L})}; (\mathcal{P}, \mathcal{L}) \text{ geometrija, ki ustreza aksiomu } \mathcal{A}\}$. Torej bo graf $\mathcal{G}$ na $n$ vozliščih imel ob optimalni izbiri $\mathcal{P}, \mathcal{L}, A \text{ in } B$ neodvisnostno število $C(n)$. Pokažimo dve trditvi glede 
števila $C(n)$:

    \begin{trditev} (Alexander Ravsky)
        $C(n) = \Omega(n^{\frac{2}{3}})$.
    \end{trditev}

    \begin{dokaz}
        Imejmo poljubno geometrijo $(\mathcal{P}, \mathcal{L})$, ki izpoljnjuje aksiom $\mathcal{A}$ in poljubni množici točk in premic $(A, B) \in \mathcal{N}_{(\mathcal{P}, \mathcal{L})}$. Označimo $a = |A|$ in $b = |B|$. Oglejmo si kartezični produkt $A \times B$, ki si ga lahko predstavljamo kot tabelo, v kateri so vrstice označene z elementi $A$ in stolpci z
        elementi $B$. Če je $p \in \mathcal{P}, L \in \mathcal{L}$ in $p \in L$, pobarvamo celico $(p, L)$, sicer pa jo pustimo nepobarvano. Ker skozi dve točki poteka kvečjemu ena premica, ne obstaja četverica paroma različnih pobarvanih parov $(p, L), (p', L), (p, L')$ in $(p', L')$, saj bi v primeru, da taka četverica obstaja,
        veljalo $L = L'$, kar je protislovje. Rečemo, da tabela nima pravokotnikov z osno-poravnanimi stranicami.

        Pokažimo zdaj splošnejšo neenakost za takšne tabele. Imejmo tabelo z $a$ vrsticami in $b$ stolpci, ki nima pravokotnikov z osno-poravninimi stranicami. Označimo njene elemente s $t_{ij}$. Tedaj ne obstaja četverica indeksov $i, j, k \text{ in } l$, da so $t_{ik}, t_{il}, t_{jk}$ in $t_{jl}$ pobarvane celice. Pravimo, da $i$-ta
        vrstica prepoveduje par stolpcev $\{j, k\}$, če sta tako $m_{ij}$ kot tudi $m_{ik}$ pobarvani. Po predhodni ugotovitvi v tabeli brez pravokotnikov z osno-poravnanimi stranicami dve različni vrstici ne moreta prepovedati istega para stolpcev. Označimo sedaj z $l_i$ število pobarvanih celic v $i$-ti vrstici. Tedaj $i$-ta vrstica
        prepove natanko $\binom{l_i}{2}$ parov stolpcev. Ker dve različni vrstici ne moreta prepovedati istega para stolpcev, z vsoto $\sum_{i=1}^a\binom{l_i}{2}$ preštejemo poljuben par stolpcev kvečjemu enkrat. Ker je vseh parov stolpcev $\binom{b}{2}$, je $\sum_{i=1}^a\binom{l_i}{2} \leq \binom{b}{2}$. To neenakost lahko nadalje razpišemo
        v $b^2 - b \geq \sum_{i=1}^{a}l_i^2 - \sum_{i=1}^{a}l_i$. Upoštevajoč Cauchy-Schwarzovo neenakost dobimo $b^2 - b \geq \frac{1}{a}(\sum_{i=1}^{a}l_i)^2 - \sum_{i=1}^{a}l_i$.

        Če se vrnemo nazaj v naš kontekst, velja $n = \sum_{i=1}^{a}l_i$ in tako $b^2 - b \geq \frac{n^2}{a} - n$ oz. $n^2 - an - a(b^2 - b) \leq 0$. Če to razrešimo kot kvadratično neenačbo sledi, da je $2n \leq a + \sqrt{a^2 + 4ab^2 - 4ab}$ oz. $(2n - a)^2 \leq a^2 + 4ab^2 - 4ab$. Od tod dobimo $n^2 - an \leq ab^2 - ab$ oz. $a \geq \frac{n^2}{b^2 - b + n}$.
        Torej je $a + b \geq \frac{n^2}{b^2 - b + n} + b$. Označimo z $b_0$ točko, v kateri je dosežen minimum desne strani neenakosti na intervalu $[1, \infty)$. Tedaj je $a + b \geq \frac{n^2}{b_0^2 - b_0 + n} + b_0 \geq b_0$. Če pišemo $f(b) = \frac{n^2}{b^2 - b + n} + b$, velja $f'(b) = \frac{b^4-2b^3+b^2(2n+1)-2bn(n+1)+2n^2}{(b^2-b+n)^2} = \frac{(b-1)(b^3-b^2+2nb-2n^2)}{(b^2-b+n)^2}$.
        Torej je $f'(b) = 0$, če je $b = 1$ ali $b^3 - b^2 + 2nb - 2n^2 = 0$. Če pišemo $g(b) = b^3 - b^2 + 2nb - 2n^2$, je $g'(b) = 3b^2 - 2b + 2n$, kar je pozitivna funkcija, torej je $g(b)$ naraščajoča kubična funkcija in zato ima enačba $b^3 - b^2 + 2nb - 2n^2 = 0$ natanko eno ničlo, ki je hkrati minimum funkcije $f$ na intervalu $[1, \infty)$.
        Ta ničla je torej ravno $b_0$ in zato velja $b_0^3 - b_0^2 + 2nb_0 - 2n^2 = 0$. Lahko preverimo, da je $g(\sqrt[3]{2n^2} - \frac{2}{3}\sqrt[3]{n}) < 0$. Zaradi naraščanja $g$ vemo, da je potem $b_0 > \sqrt[3]{2n^2} - \frac{2}{3}\sqrt[3]{n}$, torej je tudi $|A| + |B| = a + b \geq b_0 > \sqrt[3]{2n^2} - \frac{2}{3}\sqrt[3]{n}$.
        Ker so bili $\mathcal{P}, \mathcal{L}, A$ in $B$ poljubni, je $C(n) > \sqrt[3]{2n^2} - \frac{2}{3}\sqrt[3]{n}$ in zato $C(n) = \Omega(n^{\frac{2}{3}})$.
    \end{dokaz}

    \begin{trditev}
        $C(n)$ je nepadajoče zaporedje.
    \end{trditev}

    \begin{dokaz}
        Denimo, da imamo geometrijo $\Gamma = (\mathcal{P}, \mathcal{L})$, ki izpolnjuje aksiom $\mathcal{A}$ in množici točk in premic $A$ in $B$, da je $\sum_{L \in B}|L_A(\Gamma)| = m \geq n$. Tedaj lahko za vsak $L \in B$ izberemo takšno podmnožico $L'_A \subseteq L_A(\Gamma)$, da bo $\sum_{L \in B}|L_A(\Gamma)\setminus L'_A| = n$. Tedaj lahko definiramo novo geometrijo $\Gamma'$, ki ima za točke in premice prav
        tako $\mathcal{P}$ in $\mathcal{L}$, vendar velja, da v tej geometriji $p \in \mathcal{P}$ leži na $L \in \mathcal{L}$ natanko tedaj ko leži na $L$ v $\Gamma$, vendar ni iz $L'_A$. Ker smo le zmanjšali incidenčno relacijo, bo tudi geometrija $\Gamma'$ izpolnjevala aksiom $\mathcal{A}$, prav tako pa bo veljalo, da je $\sum_{L \in B}|L_A(\Gamma')| = \sum_{L \in B}|L_A(\Gamma)\setminus L'_A| = n$. 
        Ker so bile $\mathcal{P}, \mathcal{L}, A$ in $B$ poljubne, je $C(m) \geq C(n)$ za poljuben $m \geq n$ in je torej zaporedje $C(n)$ res nepadajoče.
    \end{dokaz}

Poglejmo zdaj poseben primer grafa $\mathcal{G}$. Graf $P_k$ zgradimo tako, da pogledamo končno projektivno ravnino reda $k$, torej projektivno ravnino z $k^2 + k + 1$ točkami. Za vozlišča vzamemo vse urejene pare točk in premic $(p, L)$, za katere velja,
da $p$ leži na $L$. Točke v ravnini opremimo s poljubno linearno urejenostjo $<$. Vozlišči $(p, L)$ in $(p', L')$ povežemo, če velja, da je $p < p'$, da sta $L$ in $L'$ različni in da $p$ leži na $L'$. 
Tedaj velja naslednja trditev:

    \begin{trditev}
        $P_k$ ne vsebuje trikotnikov. Velja $\alpha(P_k) \leq 2 \cdot (k^2 + k + 1)$ in $\chi(P_k) \geq \frac{k + 1}{2}$.
    \end{trditev}

    \begin{dokaz}
        $P_k$ je poseben primer grafa $\mathcal{G}$, kjer smo za geometrijo $\Gamma$ vzeli končno projektivno ravnino reda $k$, za $A$ in $B$ pa smo vzeli vse točke in premice v njej. Tedaj je $m = 3$, saj je v projektivni
        ravnini premica natanko določena z dvema točkama. Torej $P_k$ po prvi izmed prejšnjih trditev za $\mathcal{G}$ ne vsebuje klike velikosti tri, tj. ne vsebuje trikotnika. Ker imamo v projektivni ravnini reda 
        $k$ točno $k^2 + k + 1$ točk in enako toliko premic, je $|A| = |B| = k^2 + k + 1$. Od tod sledi po drugi trditvi za $\mathcal{G}$, da je $\alpha(P_k) \leq 2 \cdot (k^2 + k + 1)$ Ker na vsaki premici leži natanko 
        $k + 1$ točk, za vsak $L \in B$ velja $|L_A(\Gamma)| = k + 1$ in tako $\sum_{L \in B}|L_A(\Gamma)| = (k + 1)(k^2 + k +1)$. Od tod sledi po tretji trditvi za $\mathcal{G}$, da je $\chi(P_k) \geq \frac{k + 1}{2}$.
    \end{dokaz}

%Graf $P_k$ zgradimo tako, da pogledamo končno projektivno ravnino reda $k$, torej projektivno ravnino z $k^2 + k + 1$ točkami. Za vozlišča vzamemo vse urejene pare točk in premic $(p, L)$, za katere velja,
%da $p$ leži na $L$. Točke v ravnini opremimo s poljubno linearno urejenostjo $<$. Vozlišči $(p, L)$ in $(p', L')$ povežemo, če velja, da je $p < p'$, da sta $p$ in $p'$ različni ter $L$ in $L'$
%različni in da $p$ leži na $L'$. Tedaj velja naslednja trditev:

%    \begin{trditev}
%        $P_k$ ne vsebuje trikotnikov.
%    \end{trditev}

%    \begin{dokaz}
%        Dokazujemo s protislovjem. Denimo, da imamo trikotnik $(p, L), (p', L')$ in $(p'', L'')$. Tedaj so vse točke in premice paroma različne. Naj bo brez škode za splošnost $p < p' < p''$. Tedaj po konstrukciji
%        velja, da $p$ leži na $L, L'$ in $L''$, $p'$ pa leži na $L'$ in $L''$. Ker je v projektivni ravnini premica natanko določena z dvema točkama, ki ležita na njej, in ker tako $p$ kot $p'$ ležita na $L'$ in $L''$,
%        je $L' = L''$, kar pa je protislovje. Torej $P_k$ res ne vsebuje trikotnikov.
%    \end{dokaz}
    
%    \begin{trditev}
%        Velja $\alpha(P_k) \leq 2 \cdot (k^2 + k + 1)$.
%    \end{trditev}

%    \begin{dokaz}
%        Naj bo $H \subseteq V(P_k)$ poljubna neodvisna množica vozlišč. Tedaj lahko $H$ zapišemo kot disjunktno unijo dveh množic.

%        Prva množica vozlišč vsebuje urejene pare točk in premic iz $H$, za katere velja, da se točka pojavi v vsaj dveh urejenih parih iz $H$. Vzemimo dva takšna urejena para, v katerih se točka ponovi, recimo $(p, L)$ in $(p, L')$.
%        Ker je množica $H$ neodvisna, mora za vsako vozlišče $(q, L') \in H, q \neq p$ veljati $q < p$, saj bi sicer imeli povezavo med $(p, L)$ in $(q, L')$, kar bi bilo v protislovju z neodvisnostjo $H$. Podobno mora 
%        za vsako vozlišče $(q, L) \in H, q \neq p$ veljati $q < p$, saj bi sicer imeli povezavo med $(p, L')$ in $(q, L)$. To pa že pomeni, da je izmed vseh točk, ki se pojavijo v paru z $L$ v množici $H$, točka $p$ maksimalna.
%        Ker je maksimum v linearni urejenosti enolično določen, se bo z vsake premice, ki se pojavi v nekem paru v $H$, kvečjemu ena točka pojavila v več kot enem paru. To pomeni, da bo v prvi množici vozlišč vsaka premica iz
%        projektivne ravnine nastopila v kvečjemu enem paru. Ker je vseh premic v ravnini $k^2 + k + 1$, bo torej vozlišč v prvi množici kvečjemu $k^2 + k + 1$.

%        Druga množica vozlišč vsebuje urejene pare točk in premic iz $H$, za katere velja, da se točka pojavi v natanko enem urejenem paru iz $H$. Vozlišč v drugi množici bo torej kvečjemu toliko, kolikor je točk v ravnini, teh pa
%        je $k^2 + k + 1$.
        
%        Množica $H$, ki je unija prve in druge množice, vsebuje torej kvečjemu $2 \cdot (k^2 + k + 1)$ vozlišč. Ker je bila $H$ poljubna neodvisna množica, smo dokazali $\alpha(G) \leq 2 \cdot (k^2 + k + 1)$.
%    \end{dokaz}
    
%    \begin{trditev}
%        Velja $\chi(P_k) \geq \frac{k + 1}{2}$.
%    \end{trditev}

%    \begin{dokaz}
%        V končni projektivni ravnini reda $k$ imamo $k^2 + k + 1$ točk in vsaka točka leži na $k + 1$ premic. Torej ima graf $P_k$ $(k^2 + k + 1)(k + 1)$ vozlišč. Če upoštevamo prejšnjo trditev in neenakost \ref{spodnja_meja}, dobimo, da je
%        res $\chi(P_k) \geq \frac{k + 1}{2}$.
%    \end{dokaz}
    
Ker ima $P_k$ $(k^2 + k + 1)(k + 1)$ vozlišč, nam prejšnje trditve povejo, da je $\chi(P_k) = \Omega(n^{\frac{1}{3}})$, kjer je $n = |V(P_k)|$ kar je občutno večje kromatično število kot v prejšnjih konstrukcijah in kaže, da je lahko graf globalno res precej kompleksen
kljub odsotnosti trikotnikov.

Na analogen način bi lahko zgradili graf $A_k$ iz končne afine geometrije reda $k$, torej končne afine geometrije s $k^2$ točkami. Le-ta ima $k^2 + k$ premic in na vsaki premici leži $k$ točk. Torej ima graf $A_k$ natanko $k^2(k + 1)$ vozlišč in na analogen način kot prejšnjo
trditev pokažemo naslednjo trditev:

    \begin{trditev}
        $A_k$ ne vsebuje trikotnikov. Velja $\alpha(A_k) \leq 2 k^2 + k$ in $\chi(P_k) \geq \frac{k(k + 1)}{2k + 1}$.
    \end{trditev}

Tudi za $A_k$ torej velja, da je $\chi(A_k) = \Omega(n^{\frac{1}{3}})$, kjer je $n = |V(A_k)|$.

To je dejansko zelo blizu največjemu kromatičnemu številu in najmanjšemu neodvisnostnemu številu, ki ga lahko dosežemo v grafu brez trikotnikov. Za vse grafe brez trikotnikov $G$ velja namreč veljata naslednji meji.

    \begin{trditev}
        V grafu $G$ brez trikotnikov na $n$ vozliščih, je $\alpha(G) \geq \lfloor\sqrt{n}\rfloor$.
    \end{trditev}

    \begin{dokaz}
        Naj bo $G$ graf brez trikotnikov. Standardno označimo maksimalno stopnjo nekega vozlišča v grafu z $\Delta(G)$. Naj ima $v$ stopnjo $\Delta(G)$. Če je $\Delta(G) \geq \lfloor\sqrt{n}\rfloor$, je po \ref{karakterizacija} množica $N(v)$ neodvisna množica z močjo vsaj 
        $\lfloor\sqrt{n}\rfloor$, torej je $\alpha(G) \geq \lfloor\sqrt{n}\rfloor$. Če je $\Delta(G) \leq \lfloor\sqrt{n}\rfloor - 1$, pa po \ref{zgornja_meja} obstaja dobro barvanje grafa z $\lfloor\sqrt{n}\rfloor$ barvami. Pri takšnem barvanju moramo vsaj eno barvo uporabiti
        na $\lfloor\sqrt{n}\rfloor$ vozliščih in ta vozlišča tvorijo neodvisno množico. Torej je spet $\alpha(G) \geq \lfloor\sqrt{n}\rfloor$.
    \end{dokaz}

    \begin{trditev}
        V grafu $G$ brez trikotnikov na $n$ vozliščih je $\chi(G) \leq 2\sqrt{n} + 1$
    \end{trditev}

    \begin{dokaz}
        Dokazujemo z indukcijo na številu vozlišč. Če je $\Delta(G) \leq \sqrt(n)$, po \ref{zgornja_meja} velja $\chi(G) \leq \sqrt{n} + 1 \leq 2\sqrt{n} + 1$. Če je $\Delta(G) \geq \sqrt{n}$ in je maksimalna stopnja dosežena v $v$, je $H = N(v)$ neodvisna in jo lahko pobarvamo z isto barvo, velja
        pa da ima $H$ vsaj $\sqrt{n}$ vozlišč. Torej je $G \setminus H$ graf brez trikotnikov na kvečjemu $n - \sqrt{n}$ vozlišč. Po indukcijski predpostavki ga lahko tedaj pobarvamo s kvečjemu $2\sqrt{n - \sqrt{n}} + 1$ barvami. Ker je $\sqrt{n - \sqrt{n}} \leq \sqrt{n} - \frac{1}{2}$, kar vidimo
        s kvadriranjem obeh strani neenakosti, lahko dejansko $G \setminus H$ pobarvamo s kvečjemu $2\sqrt{n}$ barvami. Kot smo že ugotovili, lahko $H$ pobarvamo z isto barvo, torej lahko $G$ pobarvamo s kvečjemu $2\sqrt{n} + 1$ barvami. S tem smo pokazali, da je res $\chi(G) \leq 2\sqrt{n} + 1$.
    \end{dokaz}

Dejansko velja za takšne grafe še strožja meja $\chi(G) \leq (2 + o(1))\sqrt{\frac{n}{\log{n}}}$ (težji dokaz v članku "The $\chi$-Ramsey problem for triangle-free graphs").

TU JE TREBA ŠE RAZLOŽITI, ZAKAJ JIH IMENUJEMO "RAMSEYJEVI GRAFI"!!!!!!!!!!!!!

\section{Grafi s poljubno veliko ožino in poljubno velikim kromatičnim številom}
V prejšnjem razdelku smo pokazali, da odsotnost trikotnikov ne omejuje kromatičnega števila grafa. Prav tako smo poskušali ugotoviti, kako globalno kompleksen je lahko graf, ki nima trikotnikov, torej graf,
ki je lokalno na poljubnih treh vozliščih izredno enostaven. V tem razdelku bomo najprej pokazali, da ožina sama po sebi ne more omejiti kromatičnega števila, tj. da za poljubno veliko fiksno ožino vedno obstaja
graf s poljubno velikim kromatičnim številom. Nato pa bomo poskusili analizirati, kako skupaj velikost grafa in njegova ožina omejujeta kromatično število. Pogledali bomo, koliko kompleksen je lahko nek graf globalno 
v odvisnosti od tega, koliko enostaven je lokalno.

Najprej definiramo pojem naključnega grafa:

    \begin{definicija}
        Verjetnostni prostor slučajnih grafov $G(n, p)$ je podan z množico vseh grafov na $n$ vozliščih kot množico elementarnih izidov, dogodek je poljubna množica grafov na $n$ vozliščih, verjetnost vsakega grafa $G$
        z $m$ povezavami pa je enaka $P(G) = p^m(1 - p)^{\binom{n}{2}-m}$.
    \end{definicija}

    \begin{opomba}
        Verjetnost posameznega grafa v prostoru slučajnih grafov lahko interpretiramo tako, da za vsako posamezno povezavo neodvisno izbiramo, ali je vsebovana v grafu, in $p$ je verjetnost, da je posamezna povezava vsebovana v grafu.
    \end{opomba}

Naslednja lema je pomembna za dokaz glavnega izreka:

    \begin{lema}
        Naj bo slučajna spremenljivka $X : G(n, p) \to \mathbb{R}$ število $k$-ciklov v $G = (V, E) \in G(n, p)$. Potem je:

        $$E(X) = \frac{n^{\underline{k}}}{2k}p^k,$$

        kjer je $n^{\underline{k}} = n(n-1)\ldots(n-k+1)$.
    \end{lema}

    \begin{dokaz}
        Preštejmo število ciklov dolžine $k$ v polnem grafu na $n = |V|$ vozliščih. Pomagamo si z zaporedji vozlišč $C = v_1,v_2,\ldots,v_k$. Število zaporedij z različnimi elementi dolžine $k$ na $n$-elementni množici je $n^{\underline{k}}$, vsak cikel
        pa ustreza $2k$ takšnim zaporedjem, saj cikel porodi $k$ možnih začetnih točk zaporedji in imamo dve možni smeri premikanja po ciklu. Število možnih ciklov je torej $\frac{n^{\underline{k}}}{2k}$. Za vsak cikel $C_i$, $1 \leq i \leq \frac{n^{\underline{k}}}{2k}$,
        definiramo indikatorsko slučajno spremenljivko $X_i$:

        $$X_i = \begin{cases}
            1, & C_i \subseteq G \\
            0, & \text{sicer}
        \end{cases}.$$

        Velja $E[X_i] = P(X_i = 1) = p^k$, saj je po zgornji opombi to verjetnost, da je vseh $k$ povezav iz $C_i$ vsebovanih v $G$. Prav tako velja, da je $X = \sum_{i=1}^{\frac{n^{\underline{k}}}{2k}}X_i$, saj oba izraza preštejeta vse cikle dolžine $k$
        v grafu $G$. Torej po linearnosti pričakovane vrednosti velja:

        $$E[X] = E[\sum_{i=1}^{\frac{n^{\underline{k}}}{2k}}X_i] = \sum_{i=1}^{\frac{n^{\underline{k}}}{2k}}E[X_i] = \sum_{i=1}^{\frac{n^{\underline{k}}}{2k}}p^k = \frac{n^{\underline{k}}}{2k}p^k.$$
    \end{dokaz}

%    \begin{trditev}
%        Naj bo $h(k, l)$ najmanjše število, za katero vsak graf z $h(k, l)$ vozlišči vsebuje cikel dolžine kvečjemu $k$ ali množico $l$ neodvisnih vozlišč. Tedaj za fiksni $k$ in dovolj velik $l$ velja $h(k, l) > l^{1+\frac{1}{k}}$.
%    \end{trditev}

%    PRED TEM JE ŠE TREBA DOKAZATI DA $h(k, l)$ OBSTAJA IN DEFINIRATI NEODVISNO MNOŽICO.

%    \begin{dokaz}
%        Naj bo $n$ veliko število. Naj bo $0 < \epsilon < \frac{1}{k}$ poljuben. Naj bo $m = \lfloor n^{1+\epsilon} \rfloor$ in $p = \lfloor n^{1-\eta} \rfloor$, kjer je $0 < \eta < \frac{\epsilon}{2}$ poljuben.
%        Naj bo $G^{(n)}$ poln graf z $n$ vozlišči $x_1, \ldots, x_n$ in $G^{(p)}$ njegov poljuben poln podgraf s $p$ vozlišči. Očitno je, da lahko $G^{(p)}$ izberemo na $\binom{n}{p}$ načinov. Naj bodo $G_{\alpha}^{(n)}$ podgrafi
%        $G^{(n)}$, ki vsebujejo $m$ povezav, pri čemer je $1 \leq \alpha \leq \binom{\binom{n}{2}}{m}$, saj imamo toliko podgrafov z $m$ povezavami.

%        Najprej bomo pokazali, da je kvečjemu $o(\binom{\binom{n}{2}}{m})$ podgrafov $G_{\alpha}^{(n)}$, ki delijo največ $n$ povezav z $G^{(p)}$. Podgrafe, ki delijo natanko $l$ povezav z $G^{(p)}$, dobimo tako, da izberemo $l$ povezav v $G^{(p)}$ in 
%        $m - l$ povezav izven $G^{(p)}$. To lahko naredimo na $\binom{\binom{p}{2}}{l}\binom{\binom{n}{2}-\binom{p}{2}}{m-l}$ načinov. Podgrafov $G_{\alpha}^{(n)}$, ki delijo največ $n$ povezav z $G^{(p)}$, je torej
        
%        $$\sum_{l=0}^{n}\binom{\binom{p}{2}}{l}\binom{\binom{n}{2}-\binom{p}{2}}{m-l}.$$

%        Ker je $\eta < \frac{\epsilon}{2} < \frac{1}{2k} \leq \frac{1}{2}$, je $1 - \eta > \frac{1}{2}$ in s tem je $p = \omega(n^{\frac{1}{2}})$. Posledično je $\binom{p}{2} = \omega(n)$, kar pomeni, da za dovolj velike
%        $n$ velja $2n < \binom{p}{2}$. Posledično za dovolj velike $n$ velja $\binom{\binom{p}{2}}{l} < \binom{\binom{p}{2}}{n}$ za vse $l \leq n$. Prav tako velja $1 + \epsilon < 1 + \frac{1}{k} \leq 2$ in s tem $m = o(n^2)$.
%        Ker je $p = o(n)$, velja, da je $\binom{n}{2}-\binom{p}{2} = \Theta(n^2)$, torej je $\binom{n}{2}-\binom{p}{2} = \omega(m)$. Po podobnem razmisleku kot prej izpeljemo, da zato za dovolj velike $n$ velja
%        $\binom{\binom{n}{2}-\binom{p}{2}}{m-l} < \binom{\binom{n}{2}-\binom{p}{2}}{m}$ za vse $l \leq n$. Posledično za dovolj velik $n$ velja

%        $$\sum_{l=0}^{n}\binom{\binom{p}{2}}{l}\binom{\binom{n}{2}-\binom{p}{2}}{m-l} < (n + 1)\binom{\binom{p}{2}}{n}\binom{\binom{n}{2}-\binom{p}{2}}{m}.$$
%    \end{dokaz}

% \section{Zaključek}
% ...

\end{document}
