\documentclass[mat1, tisk]{fmfdelo}
% \documentclass[fin1, tisk]{fmfdelo}
% Če pobrišete možnost tisk, bodo povezave obarvane,
% na začetku pa ne bo praznih strani po naslovu, …

%%%%%%%%%%%%%%%%%%%%%%%%%%%%%%%%%%%%%%%%%%%%%%%%%%%%%%%%%%%%%%%%%%%%%%%%%%%%%%%
% METAPODATKI
%%%%%%%%%%%%%%%%%%%%%%%%%%%%%%%%%%%%%%%%%%%%%%%%%%%%%%%%%%%%%%%%%%%%%%%%%%%%%%%

% - vaše ime
\avtor{Matija Kocbek}

% - naslov dela v slovenščini
\naslov{Redkejši grafi z velikim kromatičnim številom}

% - naslov dela v angleščini
\title{Sparse graphs with high chromatic number}

% - ime mentorja/mentorice s polnim nazivom:
%   - doc.~dr.~Ime Priimek
%   - izr.~prof.~dr.~Ime Priimek
%   - prof.~dr.~Ime Priimek
%   za druge variante uporabite ustrezne ukaze
\mentor{prof.~dr.~Riste Škrekovski}
% \somentor{...}
% \mentorica{...}
% \somentorica{...}
% \mentorja{...}{...}
% \somentorja{...}{...}
% \mentorici{...}{...}
% \somentorici{...}{...}

% - leto diplome
\letnica{2025} 

% - povzetek v slovenščini
%   V povzetku na kratko opišite vsebinske rezultate dela. Sem ne sodi razlaga
%   organizacije dela, torej v katerem razdelku je kaj, pač pa le opis vsebine.
\povzetek{...}

% - povzetek v angleščini
\abstract{...}

% - klasifikacijske oznake, ločene z vejicami
%   Oznake, ki opisujejo področje dela, so dostopne na strani https://www.ams.org/msc/
\klasifikacija{..., ...}

% - ključne besede, ki nastopajo v delu, ločene s \sep
\kljucnebesede{...\sep ...}

% - angleški prevod ključnih besed
\keywords{...\sep ...} % angleški prevod ključnih besed

% - angleško-slovenski slovar strokovnih izrazov
\slovar{
% \geslo{angleški izraz}{slovenski izraz}
% ...
}

% - ime datoteke z viri (vključno s končnico .bib), če uporabljate BibTeX
% \literatura{....bib}

%%%%%%%%%%%%%%%%%%%%%%%%%%%%%%%%%%%%%%%%%%%%%%%%%%%%%%%%%%%%%%%%%%%%%%%%%%%%%%%
% DODATNE DEFINICIJE
%%%%%%%%%%%%%%%%%%%%%%%%%%%%%%%%%%%%%%%%%%%%%%%%%%%%%%%%%%%%%%%%%%%%%%%%%%%%%%%

% naložite dodatne pakete, ki jih potrebujete
% \usepackage{...}

% deklarirajte vse matematične operatorje, da jih bo LaTeX pravilno stavil
% \DeclareMathOperator{\...}{...}

% vstavite svoje definicije ...
% \newcommand{\...}{...}

%%%%%%%%%%%%%%%%%%%%%%%%%%%%%%%%%%%%%%%%%%%%%%%%%%%%%%%%%%%%%%%%%%%%%%%%%%%%%%%
% ZAČETEK VSEBINE
%%%%%%%%%%%%%%%%%%%%%%%%%%%%%%%%%%%%%%%%%%%%%%%%%%%%%%%%%%%%%%%%%%%%%%%%%%%%%%%

\begin{document}

\section{Uvod}
Grafi ponazarjajo relacije med različnimi objekti. Ena od ključnih lastnosti za razumevanje grafa je, koliko so vozlišča v njem "soodvisna". 
Kot eno izmed glavnih mer "soodvisnosti" vozlišč lahko vzamemo kromatično število grafa. 

    \begin{definicija}
        Naj bo $G = (V, E)$ (neusmerjen) graf. Naj bo $K$ poljubna neprazna množica moči $k$. Tedaj preslikavi $c: V \to K$ pravimo 
        $k$-barvanje vozlišč grafa $G$. Če sta poljubni sosednji vozlišči v $G$ pobarvani z različnimi barvami, tj. če za sosednji 
        $u$ in $v$ velja, da je $c(u) \neq c(v)$, potem pravimo, da je takšno $k$-barvanje dobro. Kromatično število $G$ je najmanjše
        število $k$, za katerega obstaja dobro $k$-barvanje grafa $G$. Označimo ga z $\chi(G)$.
    \end{definicija}

Tema tega diplomskega dela so grafi, ki imajo veliko ožino tj. velikost najmanjšega cikla. 

    \begin{definicija}
        Dolžini največjega cikla v grafu $G$ pravimo ožina in jo označimo z $girth(G)$.
    \end{definicija}

Takšni grafi so lokalno izredno enostavni. Če je ožina grafa $g$, bo podgraf porojen s poljubnih $g - 1$ vozlišč dejansko gozd, ker ne more 
imeti ciklov. Pokazali pa bomo, da so lahko takšni grafi globalno zelo kompleksni, če za mero kompleksnosti vzamemo kromatično število. 
Videli bomo, da ima lahko graf s poljubno veliko ožino prav tako poljubno veliko kromatično število. Razvoj razumevanja takšnih grafov 
bomo predstavili skozi primere.

\section{Grafi brez trikotnikov s poljubno velikim kromatičnim številom}
Najmanjši možen cikel v grafu je cikel dolžine tri in takšnim ciklom pravimo trikotniki. Če nas zanima, ali lahko graf s poljubno veliko
ožino ima poljubno veliko kromatično število, moramo začeti na prvem koraku in se vprašati, ali lahko ima sploh graf brez trikotnikov
poljubno veliko ožino. Pokazali bomo, da lahko, s tem da bomo predstavili nekaj konstrukcij takšnih grafov.

\subsection{Tuttejeva konstrukcija}
Prvi, ki je pokazal, da obstajajo grafi brez trikotnikov, ki imajo poljubno veliko kromatično število, je bil William Thomas Tutte, ki je
pisal pod psevdonimom Blanche Descartes. Podal je sledečo induktivno konstrukcijo.

Indukcijo delamo na kromatičnem številu $k$. Začnemo z grafom $G_1$, ki vsebuje samo eno vozlišče. Denimo sedaj, da poznamo graf $G_k$, 
ki ima $n$ vozlišč. Graf $G_{k+1}$ zgradimo tako, da vzamemo množico $Y$ z $k(n - 1) + 1$ vozlišči in brez kakršnihkoli povezav med njimi. 
Za vsak $X \subseteq Y$, ki vsebuje $n$ vozlišč vzamemo kopijo grafa $G_k$, ki jo označimo z $G_X$ in povežemo to kopijo z $X$ tako, da je 
vsako vozlišče iz $X$ povezano z natanko enim vozliščem iz $G_X$. Različnih kopij grafa $G_k$ med seboj ne povezujemo, tj. med nobenima 
vozliščema iz $G_X$ in $G_{X'}$ ne obstaja povezava, če je $X \neq X'$. S tem smo zgradili $G_{k+1}$.

Pokažimo, da je $\chi(G_k) = k$ in da $G_k$ ne vsebuje trikotnikov.

    \begin{trditev}
        Graf $G_k$ iz Tuttejeve konstrukcije ne vsebuje trikotnikov in velja $\chi(G_k) = k$.
    \end{trditev}
    
    \begin{dokaz}
        Dokazujemo z indukcijo. Graf z enim vozliščem ima kromatično število $1$ in je brez trikotnikov, torej je baza indukcije izpoljnena. 
        Denimo, da trditev za $G_k$ in pokažimo, da velja za $G_{k+1}$.

        Najprej s protislovjem pokažimo, da je $\chi(G_{k+1}) \geq k + 1$. Denimo nasprotno, da je $\chi(G_{k+1}) \leq k$. Tedaj obstaja $c$, 
        ki je pravilno $k$-barvanje grafa $g_{k+1}$. Recimo, da $G_k$ ima $n$ vozlišč. Tedaj po konstrukciji velja, da ima množica $Y$ iz 
        konstrukcije $k(n - 1) + 1$ vozlišč, ki so pobarvana pravilno z večjemu $k$ različnimi barvami. Zato mora obstajati vsaj ena barva 
        $b$, s katero je pobarvanih vsaj $n$ vozlišč iz $Y$. Vzemimo poljubnih $n$ vozlišč iz $Y$ pobarvanih s $b$ in označimo to množico 
        z $X$. Ker je po indukcijski predpostavki $\chi(G_k) \geq k$, bo $c$ pobarval $G_X$, ki je kopija $G_k$ z natanko $k$ barvami, saj 
        je $c$ pravilno barvanje. Ker je vsako vozlišče v $X$ povezano z natanko enim v $G_X$ in ker je $c$ pravilno barvanje, je $b$ različna
        barva od vseh $k$ barv, s katerimi smo pobarvali $G_X$. Torej je $c$ pobarval $G_{k+1}$ z vsaj $k + 1$ barvami, kar pa je protislovje.

        Prav tako je $\chi(G_{k+1}) \leq k + 1$. Po indukcijski predpostavki lahko namreč vsako kopijo $G_k$ pobarvamo s $k$ barvami, saj različne
        kopije niso med seboj povezane. Tedaj lahko vzamemo poljubno novo barvo, ki je nismo uporabili za kopije $G_k$ in z njo pobarvamo $Y$,
        saj elementi $Y$ nimajo med seboj povezav. S tem smo dokazali želeno, saj smo dobili pravilno $(k+1)$-barvanje grafa, in je res 
        $\chi(G_{k+1}) = k + 1$.

        Denimo, da v $G_{k+1}$ obstaja trikotnik. V trikotniku so vsa vozlišča paroma povezana, zato lahko v njem leži kvečjemu eno vozlišče iz $Y$,
        saj med vozlišči v $Y$ ni povezav. Prav tako preostali dve vozlišči morata ležati v isti kopiji $G_k$, saj nimamo povezav med različnimi kopijami.
        To pa pomeni, da imamo vozlišče v $Y$, ki je povezano z dvema različnima vozliščema iz iste kopije $G_k$, kar pa je v protislovju s konstrukcijo.
        Torej $G_{k+1}$ res ne vsebuje trikotnikov.

    \end{dokaz}

    \begin{opomba}
        Če definiramo $G_k$ le za $k \geq 3$ in za $G_3$ vzamemo cikel dolžine sedem, ob zgornji trditvi velja celo, da je $girth(G_k) \geq 6$ za vsak 
        $k \geq 3$. DOKAŽI TO!!!!!!!!!
    \end{opomba}

Tuttejeva konstrukcija torej res dokazuje, da lahko imajo grafi brez trikotnikov poljubno veliko kromatično število. Vendar so grafi v Tuttejevi konstrukciji
izredno veliki (če ima $G_k$ recimo $n$ vozlišč, ima $G_{k+1}$ potem $\binom{k(n-1)+1}{n}n + k(n-1) + 1$ vozlišč) in je v resnici kromatično število zelo majhno 
v razmerju s številom vozlišč. Iz tega aspekta je bolj zanimiva konstrukcija Mycielskega. Glavna prednost Tuttejeve konstrukcije je ta, da dokazuje, da lahko imajo 
tudi grafi brez ciklov dolžine manjše ali enake $4$ ali $5$ poljubno velika kromatična števila.

\subsection{Konstrukcija Mycielskega}
Jan Mycielski je leta 1955 podal konstrukcijo, ki iz začetnega grafa z $n$ vozlišči zgradi graf z $2n + 1$ vozlišči, ki ima večje kromatično število kot
začetni graf, hkrati pa nima trikotnikov, če jih začetni graf nima. Konstrukcija je podana na naslednji način.

Denimo, da imamo graf $G$ na $n$ vozliščih ${v_1, \ldots, v_n}$. Potem definiramo $M(G)$ kot graf z $2n + 1$ vozlišči ${a_1, \ldots, a_n, b_1, \ldots, b_n, c}$. 
Za vse $i, j$, za katere obstaja povezava $v_iv_j$ v $G$, tvorimo povezave $a_ia_j, a_ib_j$ in $a_jb_i$ v $M(G)$. Ob tem za vsak $i$ med $1$ in $n$ tvorimo 
povezavo $b_ic$ v $M(G)$. Takšnemu grafu $M(G)$ pravimo graf Mycielskega grafa $G$.

    \begin{trditev}
        Če graf $G$ nima trikotnikov, potem nima trikotnikov niti njegov graf Mycielskega $M(G)$.
    \end{trditev}

    \begin{dokaz}
        Naj bo $G$ brez trikotnikov. Dokazujemo s protislovjem. Denimo, da ima $M(G)$ nek trikotnik. Vsa vozlišča znotraj trikotnika so med seboj povezana. Ker v $M(G)$ ni povezav med 
        $b_i$ in $b_j$ za nobena $i$ in $j$, je lahko v ciklu kvečjemu eno vozlišče oblike $b_i$. To pomeni, da mora biti vsaj eno vozlišče oblike $a_i$ v trikotniku.
        Ker $c$ ni povezan z nobenim vozliščem te oblike, $c$ ne more biti v trikotniku. Torej imamo v trikotniku $a_j$ in $a_k$ za neka različna $j$ in $k$. Če bi imeli v
        trikotniku še vozlišče $a_i$ za nek $i$ različen od $j$ in $k$, bi to pomenilo, da imamo trikotnik v $G$, saj je podgraf $M(G)$ porojen z vozlišči ${a_1, \ldots, a_n}$
        izomorfen $G$ po konstrukciji, kar je protislovje. Torej je v trikotniku še vozlišče $b_i$ za nek $i$. To pa po konstrukciji $M(G)$ pomeni, da imamo v $G$ povezave $v_iv_k$, $v_iv_j$
        in $v_jv_k$, kar je trikotnik. Ker $G$ po predpostavki nima trikotnikov, smo prišli do protislovja.
    \end{dokaz}

    \begin{trditev}
        Velja $\chi(M(G)) = \chi(G) + 1$.
    \end{trditev}

    \begin{dokaz}
        Pokažimo najprej, da je $\chi(M(G)) \geq \chi(G) + 1$. Dokazujemo s protislovjem. Denimo, da je $\chi(M(G)) \leq \chi(G) = k$. Tedaj obstaja pravilno $k$-barvanje $M(G)$, recimo mu $f$.
        Brez škode za splošnost je $f(c) = k$. Zaradi pravilnosti $f$, ni nobeno vozlišče oblike $b_i$ pobarvano s $k$. Barvanje $f$ porodi pravilno $k$-barvanje grafa $G$, recimo mu $g$, podano 
        z $g(v_i) = f(a_i)$. Če je kakšno vozlišče $v_i$ v $G$ pobarvano s $k$, lahko spremenimo barvo v $f(b_i)$ in je barvanje grafa $G$ še vedno pravilno. Namreč, če imamo povezavo $v_iv_j$ v $G$, 
        imamo tudi povezavo $b_ia_j$ v $M(G)$, kar pomeni, da je $f(b_i) \neq f(a_j) = g(a_j)$ zaradi pravilnosti barvanja $f$. Torej tudi v spremenjenem barvanju nimamo nobenih sosedov z enako barvo,
        torej je to pravilno barvanje $G$. Vsa vozlišča v $G$, ki so bila pobarvana s $k$, smo na novo pobarvali z neko barvo iz $\{1, \ldots, k-1\}$, ker je $f(b_i) \neq k$ za vse $i$. To pa pomeni,
        da smo našli pravilno $(k-1)$-barvanje $G$, kar je v protislovju z $\chi(G) = k$.
        Dokažimo še $\chi(M(G)) \leq \chi(G) + 1$. Naj bo $k = \chi(G)$ in naj bo $g$ pravilno $k$-barvanje grafa $G$. Definiramo potem $f$ kot $(k+1)$-barvanje grafa $M(G)$. Naj bo $f(a_i) = f(b_i) = g(v_i)$
        za vse $i$. Naj bo $f(c) = k + 1$. Pokažimo, da je $f$ pravilno barvanje. Ker je $f(b_i) \neq k + 1$ za vse $i$, $c$ nima enake barve z nobenim sosedom. Če pa imamo v $M(G)$ povezavo oblike $a_ia_j$ ali 
        pa $a_ib_j$ za neka različna $i$ in $j$, vemo, da imamo v $G$ povezavo $v_iv_j$. Ker je $f(a_i) = g(v_i) \neq g(v_j) = f(a_j) = f(b_j)$, pri čemer smo upoštevali pravilnost $g$, vemo, da niti sosedi oblike
        $a_i$ in $a_j$ ali $a_i$ in $b_j$ ne bodo imeli enake barve. Torej je $f$ res pravilno $(k+1)$-barvanje $M(G)$ in je res $\chi(M(G)) = \chi(G) + 1$.
    \end{dokaz}

    \begin{posledica}
        Naj bo $G_3$ cikel dolžine $5$. Naj bo $G_{k+1} = M(G_k)$ za vsak $k \geq 3$. Tedaj je $G_k$ brez trikotnikov in $\chi(G_k) = k$ za vsak $k \geq 3$.
    \end{posledica}

    \begin{dokaz}
        Dokažemo z indukcijo. $G_3$ je cikel lihe dolžine, torej ima kromatično število $3$, in nima trikotnikov, saj cikel dolžine $3$. Baza indukcije je s tem izpolnjena. Indukcijski korak dokažemo enostavno 
        s prejšnjima trditvama.
    \end{dokaz}

Konstrukcija Mycielskega nam torej porodi še eno družino grafov brez trikotnikov, katerih kromatično število neomejeno narašča. Grafi iz te družine so dosti manjši kot grafi iz Tuttejeve konstrukcije,
in sicer bi z indukcijo lahko enostavno dokazali, da je število vozlišč v $G_k$ enako $3 \cdot 2^{k-2} - 1$ za $k \geq 3$. Torej kromatično število narašča logaritemsko glede na število vozlišč. To je še vedno dosti
majhno, vendar veliko boljše kot v Tuttejevi konstrukciji.

\subsection{Zamični grafi}

\subsection{Ramseyjevi grafi}
Vsi grafi iz prejšnjih podrazdelkov so dejansko imeli precej majhna kromatična števila glede na velikosti grafov. V tem podrazdelku pa bomo obravnavali grafe z dosti večjim kromatičnim številom,
torej globalno dosti bolj kompleksne grafe.

Graf $G_k$ zgradimo tako, da pogledamo končno projektivno ravnino reda $k$, torej projektivno ravnino s $k^2 + k + 1$ točkami. Za vozlišča vzamemo vse urejene pare točk in premic $(p, L)$, za katere velja,
da $p$ leži na $L$. Vozlišča opremimo s poljubno linearno urejenostjo $<$. Vozlišči $(p, L)$ in $(p', L')$ povežemo, če velja, da je $(p, L) < (p', L')$, da sta $p$ in $p'$ različni ter $L$ in $L'$
različni in da $p$ leži na $L'$. Tedaj velja, da $G_k$ ne vsebuje trikotnikov in da je $\alpha(G_k) \leq 2 \cdot (k^2 + k + 1)$. Ker ima graf $(k^2 + k + 1)(k + 1)$ vozlišč, je posledično $\chi(G_k) \geq k + 1$.
Če ima $G_k$ $n$ vozlišč, je torej $\chi(G_k) = \Omega(n^{\frac{1}{3}})$, kar je občutno večje kromatično število kot v prejšnjih konstrukcijah in kaže, da je lahko graf globalno res precej kompleksen
kljub odsotnosti trikotnikov.

To je dejansko zelo blizu največjemu kromatičnemu številu, ki ga lahko dosežemo v grafu brez trikotnikov. Za vse grafe brez trikotnikov $G$ velja namreč meja $\chi(G) \leq 2\sqrt{n} + 1$ (lažji dokaz na
Mathematics stack exchange). Dejansko velja za takšne grafe še strožja meja $\chi(G) \leq (2 + o(1))\sqrt{\frac{n}{\log{n}}}$ (težji dokaz v članku "The $\chi$-Ramsey problem for triangle-free graphs").

TU JE TREBA ŠE VELIKO DODELATI!!!!!!!!!!!!!

\section{Grafi s poljubno veliko ožino in poljubno velikim kromatičnim številom}
V prejšnjem razdelku smo pokazali, da odsotnost trikotnikov ne omejuje kromatičnega števila grafa. Prav tako smo poskušali ugotoviti, kako globalno kompleksen je lahko graf, ki nima trikotnikov, torej graf,
ki je lokalno na poljubnih treh vozliščih izredno enostaven. V tem razdelku bomo najprej pokazali, da ožina sama po sebi ne more omejiti kromatičnega števila, tj. da za poljubno veliko fiksno ožino vedno obstaja
graf s poljubno velikim kromatičnim številom. Nato pa bomo poskusili analizirati, kako skupaj velikost grafa in njegova ožina omejujeta kromatično število. Pogledali bomo, koliko kompleksen je lahko nek graf globalno 
v odvisnosti od tega, koliko enostaven je lokalno.

S pomočjo naslednje trditve bomo dokazali, da za poljubno veliko fiksno ožino grafa, obstaja družina grafov s to ožino in neomejenim kromatičnim številom.

    \begin{trditev}
        Naj bo $h(k, l)$ najmanjše število, za katero vsak graf z $h(k, l)$ vozlišči vsebuje cikel dolžine kvečjemu $k$ ali množico $l$ neodvisnih vozlišč. Tedaj za fiksni $k$ in dovolj velik $l$ velja $h(k, l) > l^{1+\frac{1}{k}}$.
    \end{trditev}

    PRED TEM JE ŠE TREBA DOKAZATI DA $h(k, l)$ OBSTAJA IN DEFINIRATI NEODVISNO MNOŽICO.

    \begin{dokaz}
        Naj bo $n$ veliko število. Naj bo $0 < \epsilon < \frac{1}{k}$ poljuben. Naj bo $m = \lfloor n^{1+\epsilon} \rfloor$ in $p = \lfloor n^{1-\eta} \rfloor$, kjer je $0 < \eta < \frac{\epsilon}{2}$ poljuben.
        Naj bo $G^{(n)}$ poln graf z $n$ vozlišči $x_1, \ldots, x_n$ in $G^{(p)}$ njegov poljuben poln podgraf s $p$ vozlišči. Očitno je, da lahko $G^{(p)}$ izberemo na $\binom{n}{p}$ načinov. Naj bodo $G_{\alpha}^{(n)}$ podgrafi
        $G^{(n)}$, ki vsebujejo $m$ povezav, pri čemer je $1 \leq \alpha \leq \binom{\binom{n}{2}}{m}$, saj imamo toliko podgrafov z $m$ povezavami.

        Najprej bomo pokazali, da je kvečjemu $o(\binom{\binom{n}{2}}{m})$ podgrafov $G_{\alpha}^{(n)}$, ki delijo največ $n$ povezav z $G^{(p)}$. Podgrafe, ki delijo natanko $l$ povezav z $G^{(p)}$, dobimo tako, da izberemo $l$ povezav v $G^{(p)}$ in 
        $m - l$ povezav izven $G^{(p)}$. To lahko naredimo na $\binom{\binom{p}{2}}{l}\binom{\binom{n}{2}-\binom{p}{2}}{m-l}$ načinov. Podgrafov $G_{\alpha}^{(n)}$, ki delijo največ $n$ povezav z $G^{(p)}$, je torej
        
        $$\sum_{l=0}^{n}\binom{\binom{p}{2}}{l}\binom{\binom{n}{2}-\binom{p}{2}}{m-l}.$$

        Ker je $\eta < \frac{\epsilon}{2} < \frac{1}{2k} \leq \frac{1}{2}$, je $1 - \eta > \frac{1}{2}$ in s tem je $p = \omega(n^{\frac{1}{2}})$. Posledično je $\binom{p}{2} = \omega(n)$, kar pomeni, da za dovolj velike
        $n$ velja $2n < \binom{p}{2}$. Posledično za dovolj velike $n$ velja $\binom{\binom{p}{2}}{l} < \binom{\binom{p}{2}}{n}$ za vse $l \leq n$. Prav tako velja $1 + \epsilon < 1 + \frac{1}{k} \leq 2$ in s tem $m = o(n^2)$.
        Ker je $p = o(n)$, velja, da je $\binom{n}{2}-\binom{p}{2} = \Theta(n^2)$, torej je $\binom{n}{2}-\binom{p}{2} = \omega(m)$. Po podobnem razmisleku kot prej izpeljemo, da zato za dovolj velike $n$ velja
        $\binom{\binom{n}{2}-\binom{p}{2}}{m-l} < \binom{\binom{n}{2}-\binom{p}{2}}{m}$ za vse $l \leq n$. Posledično za dovolj velik $n$ velja

        $$\sum_{l=0}^{n}\binom{\binom{p}{2}}{l}\binom{\binom{n}{2}-\binom{p}{2}}{m-l} < (n + 1)\binom{\binom{p}{2}}{n}\binom{\binom{n}{2}-\binom{p}{2}}{m}.$$
    \end{dokaz}

% \section{Zaključek}
% ...

\end{document}
