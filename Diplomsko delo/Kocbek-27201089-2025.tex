\documentclass[mat1, tisk]{fmfdelo}
% \documentclass[fin1, tisk]{fmfdelo}
% Če pobrišete možnost tisk, bodo povezave obarvane,
% na začetku pa ne bo praznih strani po naslovu, …

%%%%%%%%%%%%%%%%%%%%%%%%%%%%%%%%%%%%%%%%%%%%%%%%%%%%%%%%%%%%%%%%%%%%%%%%%%%%%%%
% METAPODATKI
%%%%%%%%%%%%%%%%%%%%%%%%%%%%%%%%%%%%%%%%%%%%%%%%%%%%%%%%%%%%%%%%%%%%%%%%%%%%%%%

% - vaše ime
\avtor{Matija Kocbek}

% - naslov dela v slovenščini
\naslov{Redkejši grafi z velikim kromatičnim številom}

% - naslov dela v angleščini
\title{Sparse graphs with high chromatic number}

% - ime mentorja/mentorice s polnim nazivom:
%   - doc.~dr.~Ime Priimek
%   - izr.~prof.~dr.~Ime Priimek
%   - prof.~dr.~Ime Priimek
%   za druge variante uporabite ustrezne ukaze
\mentor{prof.~dr.~Riste Škrekovski}
% \somentor{...}
% \mentorica{...}
% \somentorica{...}
% \mentorja{...}{...}
% \somentorja{...}{...}
% \mentorici{...}{...}
% \somentorici{...}{...}

% - leto diplome
\letnica{2025} 

% - povzetek v slovenščini
%   V povzetku na kratko opišite vsebinske rezultate dela. Sem ne sodi razlaga
%   organizacije dela, torej v katerem razdelku je kaj, pač pa le opis vsebine.
\povzetek{...}

% - povzetek v angleščini
\abstract{...}

% - klasifikacijske oznake, ločene z vejicami
%   Oznake, ki opisujejo področje dela, so dostopne na strani https://www.ams.org/msc/
\klasifikacija{..., ...}

% - ključne besede, ki nastopajo v delu, ločene s \sep
\kljucnebesede{...\sep ...}

% - angleški prevod ključnih besed
\keywords{...\sep ...} % angleški prevod ključnih besed

% - angleško-slovenski slovar strokovnih izrazov
\slovar{
% \geslo{angleški izraz}{slovenski izraz}
% ...
}

% - ime datoteke z viri (vključno s končnico .bib), če uporabljate BibTeX
% \literatura{....bib}

%%%%%%%%%%%%%%%%%%%%%%%%%%%%%%%%%%%%%%%%%%%%%%%%%%%%%%%%%%%%%%%%%%%%%%%%%%%%%%%
% DODATNE DEFINICIJE
%%%%%%%%%%%%%%%%%%%%%%%%%%%%%%%%%%%%%%%%%%%%%%%%%%%%%%%%%%%%%%%%%%%%%%%%%%%%%%%

% naložite dodatne pakete, ki jih potrebujete
% \usepackage{...}

% deklarirajte vse matematične operatorje, da jih bo LaTeX pravilno stavil
% \DeclareMathOperator{\...}{...}

% vstavite svoje definicije ...
% \newcommand{\...}{...}

%%%%%%%%%%%%%%%%%%%%%%%%%%%%%%%%%%%%%%%%%%%%%%%%%%%%%%%%%%%%%%%%%%%%%%%%%%%%%%%
% ZAČETEK VSEBINE
%%%%%%%%%%%%%%%%%%%%%%%%%%%%%%%%%%%%%%%%%%%%%%%%%%%%%%%%%%%%%%%%%%%%%%%%%%%%%%%

\begin{document}

\section{Uvod}
Grafi ponazarjajo relacije med različnimi objekti. Ena od ključnih lastnosti za razumevanje grafa je, koliko so vozlišča v njem "soodvisna". 
Kot eno izmed glavnih mer "soodvisnosti" vozlišč lahko vzamemo kromatično število grafa. 

    \begin{definicija}
        Naj bo $G = (V, E)$ (neusmerjen) graf. Naj bo $K$ poljubna neprazna množica moči $k$. Tedaj preslikavi $c: V \to K$ pravimo 
        $k$-barvanje vozlišč grafa $G$. Če sta poljubni sosednji vozlišči v $G$ pobarvani z različnimi barvami, tj. če za sosednji 
        $u$ in $v$ velja, da je $c(u) \neq c(v)$, potem pravimo, da je takšno $k$-barvanje dobro. Kromatično število $G$ je najmanjše
        število $k$, za katerega obstaja dobro $k$-barvanje grafa $G$. Označimo ga z $\chi(G)$.
    \end{definicija}

Tema tega diplomskega dela so grafi, ki imajo veliko ožino tj. velikost najmanjšega cikla. 

    \begin{definicija}
        Dolžini največjega cikla v grafu $G$ pravimo ožina in jo označimo z $girth(G)$.
    \end{definicija}

Takšni grafi so lokalno izredno enostavni. Če je ožina grafa $g$, bo podgraf porojen s poljubnih $g - 1$ vozlišč dejansko gozd, ker ne more 
imeti ciklov. Pokazali pa bomo, da so lahko takšni grafi globalno zelo kompleksni, če za mero kompleksnosti vzamemo kromatično število. 
Videli bomo, da ima lahko graf s poljubno veliko ožino prav tako poljubno veliko kromatično število. Razvoj razumevanja takšnih grafov 
bomo predstavili skozi primere.

\section{Grafi brez trikotnikov s poljubno velikim kromatičnim številom}
Najmanjši možen cikel v grafu je cikel dolžine tri in takšnim ciklom pravimo trikotniki. Če nas zanima, ali lahko graf s poljubno veliko
ožino ima poljubno veliko kromatično število, moramo začeti na prvem koraku in se vprašati, ali lahko ima sploh graf brez trikotnikov
poljubno veliko ožino. Pokazali bomo, da lahko, s tem da bomo predstavili nekaj konstrukcij takšnih grafov.

\subsection{Tuttejeva konstrukcija}
Prvi, ki je pokazal, da obstajajo grafi brez trikotnikov, ki imajo poljubno veliko kromatično število, je bil William Thomas Tutte, ki je
pisal pod psevdonimom Blanche Descartes. Podal je sledečo induktivno konstrukcijo.

Indukcijo delamo na kromatičnem številu $k$. Začnemo z grafom $G_1$, ki vsebuje samo eno vozlišče. Denimo sedaj, da poznamo graf $G_k$, 
ki ima $n$ vozlišč. Graf $G_{k+1}$ zgradimo tako, da vzamemo množico $Y$ z $k(n - 1) + 1$ vozlišči in brez kakršnihkoli povezav med njimi. 
Za vsak $X \subseteq Y$, ki vsebuje $n$ vozlišč vzamemo kopijo grafa $G_k$, ki jo označimo z $G_X$ in povežemo to kopijo z $X$ tako, da je 
vsako vozlišče iz $X$ povezano z natanko enim vozliščem iz $G_X$. Različnih kopij grafa $G_k$ med seboj ne povezujemo, tj. med nobenima 
vozliščema iz $G_X in G_{X'}$ ne obstaja povezava, če je $X \neq X'$. S tem smo zgradili $G_{k+1}$.

Pokažimo, da je $\chi(G_k) = k$ in da $G_k$ ne vsebuje trikotnikov.

    \begin{trditev}
        Graf $G_k$ iz Tuttejeve konstrukcije ne vsebuje trikotnikov in velja $\chi(G_k) = k$.
    \end{trditev}
    
    \begin{dokaz}
        Dokazujemo z indukcijo. Graf z enim vozliščem ima kromatično število $1$ in je brez trikotnikov, torej je baza indukcije izpoljnena. 
        Denimo, da trditev za $G_k$ in pokažimo, da velja za $G_{k+1}$.

        Najprej s protislovjem pokažimo, da je $\chi(G_{k+1}) \geq k + 1$. Denimo nasprotno, da je $\chi(G_{k+1}) \leq k$. Tedaj obstaja $c$, 
        ki je pravilno $k$-barvanje grafa $g_{k+1}$. Recimo, da $G_k$ ima $n$ vozlišč. Tedaj po konstrukciji velja, da ima množica $Y$ iz 
        konstrukcije $k(n - 1) + 1$ vozlišč, ki so pobarvana pravilno z večjemu $k$ različnimi barvami. Zato mora obstajati vsaj ena barva 
        $b$, s katero je pobarvanih vsaj $n$ vozlišč iz $Y$. Vzemimo poljubnih $n$ vozlišč iz $Y$ pobarvanih s $b$ in označimo to množico 
        z $X$. Ker je po indukcijski predpostavki $\chi(G_k) \geq k$, bo $c$ pobarval $G_X$, ki je kopija $G_k$ z natanko $k$ barvami, saj 
        je $c$ pravilno barvanje. Ker je vsako vozlišče v $X$ povezano z natanko enim v $G_X$ in ker je $c$ pravilno barvanje, je $b$ različna
        barva od vseh $k$ barv, s katerimi smo pobarvali $G_X$. Torej je $c$ pobarval $G_{k+1}$ z vsaj $k + 1$ barvami, kar pa je protislovje.

        Prav tako je $\chi(G_{k+1}) \leq k + 1$. Po indukcijski predpostavki lahko namreč vsako kopijo $G_k$ pobarvamo s $k$ barvami, saj različne
        kopije niso med seboj povezane. Tedaj lahko vzamemo poljubno novo barvo, ki je nismo uporabili za kopije $G_k$ in z njo pobarvamo $Y$,
        saj elementi $Y$ nimajo med seboj povezav. S tem smo dokazali želeno, saj smo dobili pravilno $(k+1)$-barvanje grafa, in je res 
        $\chi(G_{k+1}) = k + 1$.

        Denimo, da v $G_{k+1}$ obstaja trikotnik. V trikotniku so vsa vozlišča paroma povezana, zato lahko v njem leži kvečjemu eno vozlišče iz $Y$,
        saj med vozlišči v $Y$ ni povezav. Prav tako preostali dve vozlišči morata ležati v isti kopiji $G_k$, saj nimamo povezav med različnimi kopijami.
        To pa pomeni, da imamo vozlišče v $Y$, ki je povezano z dvema različnima vozliščema iz iste kopije $G_k$, kar pa je v protislovju s konstrukcijo.
        Torej $G_{k+1}$ res ne vsebuje trikotnikov.

    \end{dokaz}

    \begin{opomba}
        Če definiramo $G_k$ le za $k \geq 3$ in za $G_3$ vzamemo cikel dolžine sedem, ob zgornji trditvi velja celo, da je $girth(G_k) \geq 6$ za vsak 
        $k \geq 3$.
    \end{opomba}

Tuttejeva konstrukcija torej res dokazuje, da lahko imajo grafi brez trikotnikov poljubno veliko kromatično število. Vendar so grafi v Tuttejevi konstrukciji
izredno veliki in je v resnici kromatično število precej majhno v razmerju s številom vozlišč. Iz tega aspekta je bolj zanimiva konstrukcija Mycielskega.

\subsection{Konstrukcija Mycielskega}
Jan Mycielski je leta 1955 podal konstrukcijo, ki iz začetnega grafa z $n$ vozlišči zgradi graf z $2n + 1$ vozlišči, ki ima večje kromatično število kot
začetni graf, hkrati pa nima trikotnikov, če jih začetni graf nima. Konstrukcija je podana na naslednji način.

Denimo, da imamo graf $G$ na $n$ vozliščih ${v_1, \ldots, v_n}$. Potem definiramo $M(G)$ kot graf z $2n + 1$ vozlišči ${a_1, \ldots, a_n, b_1, \ldots, b_n, c}$. 
Za vse $i, j$, za katere obstaja povezava $v_iv_j$ v $G$, tvorimo povezave $a_ia_j, a_ib_j$ in $a_jb_i$ v $M(G)$. Ob tem za vsak $i$ med $1$ in $n$ tvorimo 
povezavo $b_ic$ v $M(G)$. Takšnemu grafu $M(G)$ pravimo graf Mycielskega grafa $G$.

    \begin{trditev}
        Če graf $G$ nima trikotnikov, potem nima trikotnikov niti njegov graf Mycielskega $M(G)$.
    \end{trditev}

    \begin{dokaz}
        Naj bo $G$ brez trikotnikov. Dokazujemo s protislovjem. Denimo, da ima $M(G)$ nek trikotnik. Vsa vozlišča znotraj trikotnika so med seboj povezana. Ker v $M(G)$ ni povezav med 
        $b_i$ in $b_j$ za nobena $i$ in $j$, je lahko v ciklu kvečjemu eno vozlišče oblike $b_i$. To pomeni, da mora biti vsaj eno vozlišče oblike $a_i$ v trikotniku.
        Ker $c$ ni povezan z nobenim vozliščem te oblike, $c$ ne more biti v trikotniku. Torej imamo v trikotniku $a_j$ in $a_k$ za neka različna $j$ in $k$. Če bi imeli v
        trikotniku še vozlišče $a_i$ za nek $i$ različen od $j$ in $k$, bi to pomenilo, da imamo trikotnik v $G$, saj je podgraf $M(G)$ porojen z vozlišči ${a_1, \ldots, a_n}$
        izomorfen $G$ po konstrukciji, kar je protislovje. Torej je v trikotniku še vozlišče $b_i$ za nek $i$. To pa po konstrukciji $M(G)$ pomeni, da imamo v $G$ povezave $v_iv_k$, $v_iv_j$
        in $v_jv_k$, kar je trikotnik. Ker $G$ po predpostavki nima trikotnikov, smo prišli do protislovja.
    \end{dokaz}

    \begin{trditev}
        Velja $\chi(M(G)) = \chi(G) + 1$.
    \end{trditev}

    \begin{dokaz}
        Pokažimo najprej, da je $\chi(M(G)) \geq \chi(G) + 1$. Dokazujemo s protislovjem. Denimo, da je $\chi(M(G)) \leq \chi(G) = k$. Tedaj obstaja pravilno $k$-barvanje $M(G)$, recimo mu $f$.
        Brez škode za splošnost je $f(c) = k$. Zaradi pravilnosti $f$, ni nobeno vozlišče oblike $b_i$ pobarvano s $k$. Barvanje $f$ porodi pravilno $k$-barvanje grafa $G$, recimo mu $g$, podano 
        z $g(v_i) = f(a_i)$. Če je kakšno vozlišče $v_i$ v $G$ pobarvano s $k$, lahko spremenimo barvo v $f(b_i)$ in je barvanje grafa $G$ še vedno pravilno. Namreč, če imamo povezavo $v_iv_j$ v $G$, 
        imamo tudi povezavo $b_ia_j$ v $M(G)$, kar pomeni, da je $f(b_i) \neq f(a_j) = g(a_j)$ zaradi pravilnosti barvanja $f$. Torej tudi v spremenjenem barvanju nimamo nobenih sosedov z enako barvo,
        torej je to pravilno barvanje $G$. Vsa vozlišča v $G$, ki so bila pobarvana s $k$, smo na novo pobarvali z neko barvo iz $\{1, \ldots, k-1\}$, ker je $f(b_i) \neq k$ za vse $i$. To pa pomeni,
        da smo našli pravilno $(k-1)$-barvanje $G$, kar je v protislovju z $\chi(G) = k$.
        Dokažimo še $\chi(M(G)) \leq \chi(G) + 1$. Naj bo $k = \chi(G)$ in naj bo $g$ pravilno $k$-barvanje grafa $G$. Definiramo potem $f$ kot $(k+1)$-barvanje grafa $M(G)$. Naj bo $f(a_i) = f(b_i) = g(v_i)$
        za vse $i$. Naj bo $f(c) = k + 1$. Pokažimo, da je $f$ pravilno barvanje. Ker je $f(b_i) \neq k + 1$ za vse $i$, $c$ nima enake barve z nobenim sosedom. Če pa imamo v $M(G)$ povezavo oblike $a_ia_j$ ali 
        pa $a_ib_j$ za neka različna $i$ in $j$, vemo, da imamo v $G$ povezavo $v_iv_j$. Ker je $f(a_i) = g(v_i) \neq g(v_j) = f(a_j) = f(b_j)$, pri čemer smo upoštevali pravilnost $g$, vemo, da niti sosedi oblike
        $a_i$ in $a_j$ ali $a_i$ in $b_j$ ne bodo imeli enake barve. Torej je $f$ res pravilno $(k+1)$-barvanje $M(G)$ in je res $\chi(M(G)) = \chi(G) + 1$.
    \end{dokaz}

% \section{Zaključek}
% ...

\end{document}
